\documentclass{article}

\usepackage{tcolorbox}
\usepackage[a4paper, left=2.5cm, right=2.5cm, bottom=2.5cm, top=3.0cm]{geometry}
\usepackage[spanish]{babel}
\tcbuselibrary{skins,breakable}
\usetikzlibrary{shadings,shadows,arrows.meta}
\usepackage{tikz}
\usepackage{forest}
\usepackage{amsmath}
\usepackage[skip=10pt plus1pt, indent=20pt]{parskip}

\newenvironment{myblock}[1]{%
	\tcolorbox[beamer,%
	noparskip,breakable,
%	colback=LightGreen,colframe=DarkGreen,%
%	colbacklower=LimeGreen!75!LightGreen,%
	title=#1]}%
	{\endtcolorbox}

\begin{document}

\section*{Ejercicios}

\subsection*{I.3 Reducir, mostrando cada paso, los siguientes términos:}

\subsubsection*{1. $\star ; \langle \star, \star \rangle$}

\[\star ; \langle \star, \star \rangle \rightarrow^{sec} \langle \star , \star \rangle
\]

\subsubsection*{2. $(\lambda x. \pi_1 x) \langle \star, \star \rangle$}

\[
		(\lambda x. \pi_1 x) \langle \star, \star \rangle \rightarrow^{\beta} \pi_1 \langle \star, \star \rangle \rightarrow^{\pi_1} \star
\]

\subsubsection*{3. match(inr$(\star), x.(\lambda y.y), y.(\lambda y.\star)$)}
\[
	\text{match}( \text{inr}(\star), x.(\lambda y.y), y. (\lambda y. \star)) \rightarrow^{\text{match}_r} (\text{inr}(\star)/y) (\lambda y. \star) = \lambda y. \star
\]

\subsection*{I.8 Demostrar el lema:}

\def\r{\longrightarrow_R}
\def\rs{\longrightarrow^{\star}_R}

\begin{myblock}{Lema 2.7}
	\begin{enumerate}
		\item Si $\r$ satisface la propiedad del diamante, entonces es Church-Rosser.
		\item Si $\r$ es Church-Rosser, entonces tiene formas normales únicas.
	\end{enumerate}
\end{myblock}

\subsubsection*{1}

Sea $\r$ que satisface la propiedad del diamante, es decir que, si $t \r r_1$ y $r \r r_2$ entonces existe un $s$ tal que $r_1 \r s$ y $r_2 \r s$. Queremos ver que si $t \rs r_1$ y $t \rs r_2$ entonces existe un $s$ tal que $r_1 \rs s$ y $r_2 \rs s$.

Sean $t, r_1, r_2$ tal que $t \rs r_1, t \rs r_2$. Si $t \rs r_1$ existe una secuencia $\alpha_1, \alpha_2, \ldots, \alpha_{k-1}$ tal que $t \r \alpha_1 \r \alpha_2 \r \ldots \r \alpha_{k-1} \r r_1$, lo mismo con $r_2$ para una secuencia $\beta_1, \ldots, \beta_{l-1}$. Notamos esto $t \r^k r_1, t \r^l r_2$. Vamos por inducción en la longitud de las secuencias.

\textbf{Hipótesis inductiva:} Si $t \r^k r_1, t \r^l r_2$ entonces existe un $s$ tal que $r_1 \r^l s, r_2 \r^k s$.

Para $l,r \leq 1$ esto es claramente válido porque es la propiedad del diamante.

El paso inductivo es para $l+1, r+1 \leq n+1$.

Sean $t,r_1,r_2$ tal que $t \r^{k+1} r_1, t \r^{l+1} r_2$. Sabemos que $t \r^k \alpha \r r_1$ y $t \r^l \beta \r r_2$. Por hipótesis inductiva tenemos que existe un $\gamma$ tal que $\alpha \r^l \gamma$ y $\beta \r^k \gamma$. Ahora también por HI sabemos que existen $\delta_1, \delta_2$ tal que $r_1 \r^l \delta_1, r_2 \r^k \delta_2, \gamma \r \delta_1, \gamma \r \delta_2$. Finalmente, por propiedad de diamante con $\gamma, \delta_1, \delta_2$ vemos que existe $s$ tal que $\delta_1 \r s, \delta_2 \r s$.

Uniendo todo esto tenemos que $r_1 \r^l \delta_1 \r s$ y $r_2 \r^k \delta_2 \r s$ por lo tanto $r_1 \r^{l+1} s$ y $r_2 \r^{k+1} s$ que era lo que buscábamos. Abajo hay un esquema de la demostración.

\begin{center}
	\begin{forest}
		for tree = {edge={dotted, -Straight Barb}}
		[$t$
			[$\alpha$, edge label = {node[midway, left, font=\scriptsize]{k}}, name=a
				[$r_1$, edge={solid}
					[$\delta_1$, name=d1, edge label = {node[midway, left, font=\scriptsize]{l}}]
				]
			]
			[, no edge[$\gamma$, name=g, no edge
					[, no edge[$s$, name=s, no edge]]
			]]
			[$\beta$, edge label = {node[midway, right, font=\scriptsize]{l}}, name=b
				[$r_2$, edge={solid}
					[$\delta_2$, name=d2, edge label = {node[midway, right, font=\scriptsize]{k}}]
				]
			]
		]
		\draw[-Straight Barb, dotted] (a) -- (g) node[midway, right, font=\scriptsize] {l};
		\draw[-Straight Barb, dotted] (b) -- (g) node[midway, left, font=\scriptsize] {k};
		\draw[-Straight Barb] (g) -- (d1);
		\draw[-Straight Barb] (g) -- (d2);
		\draw[-Straight Barb] (d1) -- (s);
		\draw[-Straight Barb] (d2) -- (s);
	\end{forest}
\end{center}

\subsubsection*{2}

Sea $\r$ Church-Rosser, $r_1, r_2$ en forma normal, y $t$ tal que $t \rs r_1, t \rs r_2$ queremos probar que $r_1 = r_2$.

Como $\r$ es Church-Rosser y $t \rs r_1, t \rs r_2$, sabemos que existe un $s$ tal que $r_1 \rs s$ y $r_2 \rs s$.

Como $r_1$ está en forma normal, no existe ningún $e$ tal que $r_1 \r e$, así que el único elemento tal que $r_1 \rs s$ es $r_1$. Por lo tanto $s = r_1$. Se puede hacer el mismo argumento con $r_2$ y llegar a que $r_2 = s = r_1$. Entonces $r_1 = r_2$.


\subsection*{III.26 Probar el Teorema 8.22}

\begin{myblock}{Teorema 8.22}
	Para todo operador densidad $\rho$ se tiene $tr(\rho^2) \leq 1$.

	Más aún, la igualdad se cumple si y solo si $\rho$ está en un estado puro.
\end{myblock}

Para la primera parte del teorema recordamos que para un conjunto de estados puros $\{(p_i, | \psi_i \rangle)\}$, el operador de densidad es $\rho = \sum_i p_i | \psi_i \rangle \langle \psi_i |$.

\def\ij{| \psi_i \rangle \langle \psi_i | \psi_j \rangle \langle \psi_j |}

\[
	tr(\rho^2) = tr((\sum_i p_i | \psi_i \rangle \langle \psi_i |)^2) = tr(\sum_i \sum_j p_i p_j \ij)
\]

Por las propiedades de la traza vemos que esto es igual a:

\[
	\sum_i \sum_j tr(p_i p_j \ij) = \sum_i \sum_j p_i p_j tr(\ij)
\]

Ahora, por el Teorema 8.12 del apunte que dice que $tr(|\psi \rangle \langle \varphi | = \langle \varphi | \psi \rangle$, tomamos $\varphi$ como $\psi_j$ y $\psi$ como el resto, y tenemos que:

\[
	tr(\ij) = \langle \psi_j | \psi_i \rangle \langle \psi_i | \psi_j \rangle = | \langle \psi_j | \psi_i \rangle |^2
\]

Por Cauchy-Schwartz, sabemos que $|\langle \psi_j | \psi_i \rangle|^2 \leq \langle \psi_j | \psi_j \rangle \langle \psi_i | \psi_i \rangle$, y, como los vectores son unitarios, tenemos que $\langle \psi_k | \psi_k \rangle = 1$ para todo $k$. Entonces:

\[
	tr(\rho^2) = \sum_i \sum_j p_i p_j |\langle \psi_j | \psi_i \rangle|^2 \leq \sum_i \sum_j p_i p_j \leq 1
\]

La última inecuación vale porque los $p_k$ son probabilidades y por lo tanto son menores o iguales a 1.

Cuando $\rho$ está en un estado puro es fácil ver que:

\[
	tr(\rho^2) = tr(|\psi \rangle \langle \psi | \psi \rangle \langle \psi |) = \langle \psi | \psi \rangle \langle \psi | \psi \rangle = 1 \cdot 1 = 1
\]

\end{document}
