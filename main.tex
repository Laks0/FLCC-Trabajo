\documentclass{article}

\usepackage{tcolorbox}
\usepackage[a4paper, left=2.5cm, right=2.5cm, bottom=2.5cm, top=3.0cm]{geometry}
\usepackage[spanish]{babel}
\tcbuselibrary{skins,breakable}
\usetikzlibrary{shadings,shadows,arrows.meta}
\usepackage{tikz}
\usepackage{forest}
\usepackage{amsmath}
\usepackage{amsfonts}
\usepackage[braket]{qcircuit}
\usepackage[skip=10pt plus1pt, indent=20pt]{parskip}

\newenvironment{myblock}[1]{%
	\tcolorbox[beamer,%
	noparskip,breakable,
%	colback=LightGreen,colframe=DarkGreen,%
%	colbacklower=LimeGreen!75!LightGreen,%
	title=#1]}%
	{\endtcolorbox}

\setlength\parindent{0pt}


\setlength{\parskip}{0.1em}
\newcommand{\tab}{~ \qquad}
\usepackage{caratula/caratula} % Version modificada para usar las macros de algo1 de ~> https://github.com/bcardiff/dc-tex

\begin{document}

\titulo{Exámen final}
\subtitulo{}
\fecha{09 de Diciembre del 2025}
\materia{Fundamentos de Lenguajes para Computación Cuántica}
%\grupo{Grupo XX}
\newcommand{\senial}{\textit{se\~nal}}


\integrante{Bagnasco Muguillo, Lautaro}{1173/21 }{lbagnasco@dc.uba.ar}
\integrante{Laks, Joaquín}{??}{??}
\integrante{Vekselman, Natan}{338/21}{natanvek11@gmail.com}
\integrante{Romani, Rafael}{775/21}{rafaromani243@gmail.com}

\maketitle



\section*{Parte I: Lógica, cálculo lambda, y el isomorfismo de Curry-Howard}
\subsection*{I.8 Demostrar el lema:}

\def\r{\longrightarrow_R}
\def\rs{\longrightarrow^{\star}_R}

\begin{myblock}{Lema 2.7}
	\begin{enumerate}
		\item Si $\r$ satisface la propiedad del diamante, entonces es Church-Rosser.
		\item Si $\r$ es Church-Rosser, entonces tiene formas normales únicas.
	\end{enumerate}
\end{myblock}

\subsubsection*{1}

Sea $\r$ que satisface la propiedad del diamante, es decir que, si $t \r r_1$ y $r \r r_2$ entonces existe un $s$ tal que $r_1 \r s$ y $r_2 \r s$. Queremos ver que si $t \rs r_1$ y $t \rs r_2$ entonces existe un $s$ tal que $r_1 \rs s$ y $r_2 \rs s$.

Sean $t, r_1, r_2$ tal que $t \rs r_1, t \rs r_2$. Si $t \rs r_1$ existe una secuencia $\alpha_1, \alpha_2, \ldots, \alpha_{k-1}$ tal que $t \r \alpha_1 \r \alpha_2 \r \ldots \r \alpha_{k-1} \r r_1$, lo mismo con $r_2$ para una secuencia $\beta_1, \ldots, \beta_{l-1}$. Notamos esto $t \r^k r_1, t \r^l r_2$. Vamos por inducción en la longitud de las secuencias.

\textbf{Hipótesis inductiva:} Si $t \r^k r_1, t \r^l r_2$ entonces existe un $s$ tal que $r_1 \r^l s, r_2 \r^k s$.

Para $l,r \leq 1$ esto es claramente válido porque es la propiedad del diamante.

El paso inductivo es para $l+1, r+1 \leq n+1$.

Sean $t,r_1,r_2$ tal que $t \r^{k+1} r_1, t \r^{l+1} r_2$. Sabemos que $t \r^k \alpha \r r_1$ y $t \r^l \beta \r r_2$. Por hipótesis inductiva tenemos que existe un $\gamma$ tal que $\alpha \r^l \gamma$ y $\beta \r^k \gamma$. Ahora también por HI sabemos que existen $\delta_1, \delta_2$ tal que $r_1 \r^l \delta_1, r_2 \r^k \delta_2, \gamma \r \delta_1, \gamma \r \delta_2$. Finalmente, por propiedad de diamante con $\gamma, \delta_1, \delta_2$ vemos que existe $s$ tal que $\delta_1 \r s, \delta_2 \r s$.

Uniendo todo esto tenemos que $r_1 \r^l \delta_1 \r s$ y $r_2 \r^k \delta_2 \r s$ por lo tanto $r_1 \r^{l+1} s$ y $r_2 \r^{k+1} s$ que era lo que buscábamos. Abajo hay un esquema de la demostración.

\begin{center}
	\begin{forest}
		for tree = {edge={dotted, -Straight Barb}}
		[$t$
			[$\alpha$, edge label = {node[midway, left, font=\scriptsize]{k}}, name=a
				[$r_1$, edge={solid}
					[$\delta_1$, name=d1, edge label = {node[midway, left, font=\scriptsize]{l}}]
				]
			]
			[, no edge[$\gamma$, name=g, no edge
					[, no edge[$s$, name=s, no edge]]
			]]
			[$\beta$, edge label = {node[midway, right, font=\scriptsize]{l}}, name=b
				[$r_2$, edge={solid}
					[$\delta_2$, name=d2, edge label = {node[midway, right, font=\scriptsize]{k}}]
				]
			]
		]
		\draw[-Straight Barb, dotted] (a) -- (g) node[midway, right, font=\scriptsize] {l};
		\draw[-Straight Barb, dotted] (b) -- (g) node[midway, left, font=\scriptsize] {k};
		\draw[-Straight Barb] (g) -- (d1);
		\draw[-Straight Barb] (g) -- (d2);
		\draw[-Straight Barb] (d1) -- (s);
		\draw[-Straight Barb] (d2) -- (s);
	\end{forest}
\end{center}

\subsubsection*{2}

Sea $\r$ Church-Rosser, $r_1, r_2$ en forma normal, y $t$ tal que $t \rs r_1, t \rs r_2$ queremos probar que $r_1 = r_2$.

Como $\r$ es Church-Rosser y $t \rs r_1, t \rs r_2$, sabemos que existe un $s$ tal que $r_1 \rs s$ y $r_2 \rs s$.

Como $r_1$ está en forma normal, no existe ningún $e$ tal que $r_1 \r e$, así que el único elemento tal que $r_1 \rs s$ es $r_1$. Por lo tanto $s = r_1$. Se puede hacer el mismo argumento con $r_2$ y llegar a que $r_2 = s = r_1$. Entonces $r_1 = r_2$.

\section*{Parte II: Semántica denotacional y teoría de categorías}
\subsection*{Ejercicio II.5}

Probar el teorema 4.12


\begin{myblock}{Teorema 4.12}
    En Set, los epimorfismos son exactamente las funciones sobreyectivas (las funciones $f : A \to B$ para las cuales para cada $b \in B$ existe un $a \in A$ tal que $f (a) = b$).
\end{myblock}

$\implies)$ Sea $A \overset{\text{f}} \longrightarrow B$ una función sobreyectiva. Supongamos que no es un epimorfismo, por lo tanto existen $B \overset{\text{g}} \longrightarrow C$ y $B \overset{\text{h}} \longrightarrow C$ tales que $g\circ f = h \circ f$ pero $g \neq h$. Sea $y \in B$ tal que $g(y) \neq h(y)$ el cual existe por ser $g$ y $h$ distintas. Como $f$ es sobreyectiva, existe $x \in A$ que cumple $f(x) = y$, luego $g(y) = g(f(y))$ y $h(y) = h(f(y))$ y por ende $g\circ f \neq h \circ f$, lo cual contradice las hipótesis de $g$ y $h$.

$\impliedby)$ Sea $A \overset{\text{f}} \longrightarrow B$ un epimorfismo. Supongamos que no es sobreyectiva, luego existe $b \in B$ tal que para todo $a \in A$, $f(a) \neq b$. Sean $g$ y $h$ flechas de $B$ en $C$ tales que 
\begin{align*}
    &g(x) = x 
    &h(x) = \begin{cases}
        c & \text{si } x = b \\
        x & \text{caso contrario}
    \end{cases}  
\end{align*}

Luego, ver que $g(f(x)) = f(x)$ y $h(f(x)) = f(x)$ pues $f(x) \neq b$ para todo $x \in A$, por lo tanto $g\circ f = h \circ f$ y dado que $f$ es un epimorfismo se tiene que $g = h$ lo cual contradice las hipótesis.
\subsection*{Ejercicio 10: demostrar que $\mathrm{List}$ es un functor}

\begin{myblock}{Comportamiento functorial de $\mathrm{List}$}
Sea $f:S \to S'$ una función entre conjuntos, probar que

\begin{enumerate}
    \item $ \mathrm{List}(\mathrm{Id}_S) = \mathrm{Id}_{\mathrm{List}(S)} $
    \item $ \mathrm{List}(g \circ f) = \mathrm{List}(g) \circ \mathrm{List}(f) $ para toda función $g:S' \to S''$.
\end{enumerate}
\end{myblock}


\begin{enumerate}
    \item \emph{Identidad.} Sea $L = [s_1, \dots, s_n] \in \mathrm{List}(S)$ una lista cualquiera, luego se tiene:
    \[
        \mathrm{List}(\mathrm{Id}_S)(L) = [\mathrm{Id}_S(s_1) , \dots , \mathrm{Id}_S(s_n)] = [s_1, \dots, s_n] = \mathrm{Id}_{\mathrm{List}(S)}(L)
    \]
    por lo tanto $\mathrm{List}(\mathrm{Id}_S) = \mathrm{Id}_{\mathrm{List}(S)}$.


    \item \emph{Composición.} Sean $f:S \to S'$, $g:S' \to S''$ y $L=[s_1, \dots, s_n]$ una lista en $\mathrm{List}(S)$. Entonces:
    \[
        \mathrm{List}(g \circ f)(L) = [(g \circ f)(s_1), \dots , (g \circ f)(s_n)] = [g(f(s_1)), \dots, g(f(s_n))]
    \]
    Luego, desde el otro lado
    \[
        [g(f(s_1)) , \dots , g(f(s_n))] = \mathrm{List}(g)([f(s_1), \dots , f(s_n)]) = (\mathrm{List}(g) \circ \mathrm{List}(f))(L)
    \]
    Dado que esto vale para toda lista se tiene que
    \[
        \mathrm{List}(g \circ f) = \mathrm{List}(g) \circ \mathrm{List}(f)
    \]
    lo cual demuestra la propiedad de composición.

\end{enumerate}

Por ende $\mathrm{List}$ cumple con las dos condiciones por lo que queda probado que es un functor.

\section*{Parte III: Computación cuántica}
\subsection*{III. 15 Dar un circuito que genere el estado $\frac{1}{\sqrt{2}}|000\rangle + \frac{1}{\sqrt{2}}|111\rangle$ a partir de la entrada $|000\rangle$}

Funciona el siguiente circuito:

\[
	\Qcircuit @R=1em @C=1em{
		\lstick{\ket{0}} & \gate{H} & \ctrl{1} & \ctrl{2} & \qw \\
		\lstick{\ket{0}} & \qw      & \targ    & \qw      & \qw \\
		\lstick{\ket{0}} & \qw      & \qw      & \targ    & \qw \\
	}
\]

Si analizamos la traza vemos que, comenzando con la entrada \ket{000} llegamos al estado deseado.

\def\hr{\frac{1}{\sqrt{2}}}

\[
	\ket{000} \xrightarrow{H\otimes I \otimes I} (\hr\ket{0}+\hr\ket{1}) \ket{00} = \hr (\ket{000} + \ket{100})
\]
\[
	\xrightarrow{CNOT(1,2)} \hr (\ket{000}+\ket{110}) \xrightarrow{CNOT(1,3)} \hr (\ket{000}+\ket{111})
\]

\subsection*{III.26 Probar el Teorema 8.22}

\begin{myblock}{Teorema 8.22}
	Para todo operador densidad $\rho$ se tiene $tr(\rho^2) \leq 1$.

	Más aún, la igualdad se cumple si y solo si $\rho$ está en un estado puro.
\end{myblock}

\def\ij{| i \rangle \langle i | j \rangle \langle j |}

Recordamos que para un conjunto de estados puros $\{(p_i, | \psi_i \rangle)\}$, el operador de densidad es $\rho = \sum_i p_i | \psi_i \rangle \langle \psi_i |$.
Sabemos por el Teorema 8.25 del apunte que lo podemos escribir como $\rho = \sum_j \lambda_j | j \rangle \langle j |$ donde los $|j\rangle$ son ortonormales y $\lambda_j \in \mathbb{R}_0^+$. Vemos entonces que:

\[
	tr(\rho^2) = tr((\sum_i \lambda_i |i\rangle \langle i|)^2) = tr(\sum_i \sum_j \lambda_i \lambda_j \ij) = \sum_i \sum_j \lambda_i \lambda_j tr(\ij)
\]

Como los vectores son ortonormales, tenemos que $\langle i | j \rangle = \delta_{ij}$, entonces:

\[
	\sum_i \sum_j \lambda_i \lambda_j tr(\ij) = \sum_i \sum_j \delta_{ij} \lambda_i \lambda_j tr(|i\rangle \langle j|) = \sum_i \lambda_i^2 tr(|i\rangle\langle i|)
\]

Ahora, por el Teorema 8.12 del apunte sabemos que $tr(|i\rangle \langle i |) = \langle i | i \rangle = 1$, entonces $tr(\rho^2) = \sum_i \lambda_i^2$.

También sabemos que:

\[
	tr(\rho) = tr(\sum_i \lambda_i | i \rangle \langle i |) = \sum_i \lambda_i = 1
\]

Como los $\lambda_i$ son positivos, esto nos dice que $0 \leq \lambda_i \leq 1$ para todo $i$, por lo tanto $\lambda_i^2 \leq \lambda_i$. Entonces:

$$tr(\rho^2) = \sum_i \lambda_i^2 \leq \sum_i \lambda_i = tr(\rho) = 1$$

En el caso en el que $\rho$ está en estado puro, se puede escribir como $|\psi\rangle \langle\psi|$, y es fácil ver que $tr(\rho^2)=tr(|\psi\rangle\langle\psi|\psi\rangle\langle\psi|)=tr(|\psi\rangle\langle\psi|)=\langle\psi|\psi\rangle=1$

Cuando $\rho$ es un estado mixto, no se puede escribir de esa forma, así que sabemos que por lo menos dos $\lambda_i$ son no nulos. Como $\sum_i \lambda_i = 1$ y $0 \leq \lambda_i \leq 1$, esto nos dice que $\lambda_i < 1$ para todo $i$. Por lo tanto $\lambda_i^2 < \lambda_i$. Entonces:

\[
	tr(\rho^2) = \sum_i \lambda_i^2 < \sum_i \lambda_i = 1
\]


\newpage
\section*{Algoritmo2}
\textbf{Caso \(b_1=0,b_2=0\).}
\[
\beta_{00}=\tfrac{1}{\sqrt{2}}(\ket{00}+\ket{11})
\]
\[
\xrightarrow{X^{0}\otimes I}
\tfrac{1}{\sqrt{2}}(\ket{00}+\ket{11})
\]
\[
\xrightarrow{Z^{0}\otimes I}
\tfrac{1}{\sqrt{2}}(\ket{00}+\ket{11})
\]
\[
\xrightarrow{\mathrm{CNOT}}
\tfrac{1}{\sqrt{2}}(\ket{00}+\ket{10})
\]
\[
=\bigl(\tfrac{\ket{0}+\ket{1}}{\sqrt{2}}\bigr)\otimes\ket{0}=\ket{+}\otimes\ket{0}
\]
\[
\xrightarrow{H\otimes I}
\ket{0}\otimes\ket{0}=\ket{00}
\]

\textbf{Caso \(b_1=0,b_2=1\).}
\[
\beta_{00}=\tfrac{1}{\sqrt{2}}(\ket{00}+\ket{11})
\]
\[
\xrightarrow{X^{1}\otimes I}
\tfrac{1}{\sqrt{2}}(\ket{10}+\ket{01})
\]
\[
\xrightarrow{Z^{0}\otimes I}
\tfrac{1}{\sqrt{2}}(\ket{10}+\ket{01})
\]
\[
\xrightarrow{\mathrm{CNOT}}
\tfrac{1}{\sqrt{2}}(\ket{11}+\ket{01})
\]
\[
=\bigl(\tfrac{\ket{0}+\ket{1}}{\sqrt{2}}\bigr)\otimes\ket{1}=\ket{+}\otimes\ket{1}
\]
\[
\xrightarrow{H\otimes I}
\ket{0}\otimes\ket{1}=\ket{01}
\]

\textbf{Caso \(b_1=1,b_2=0\).}
\[
\beta_{00}=\tfrac{1}{\sqrt{2}}(\ket{00}+\ket{11})
\]
\[
\xrightarrow{X^{0}\otimes I}
\tfrac{1}{\sqrt{2}}(\ket{00}+\ket{11})
\]
\[
\xrightarrow{Z^{1}\otimes I}
\tfrac{1}{\sqrt{2}}(\ket{00}-\ket{11})
\]
\[
\xrightarrow{\mathrm{CNOT}}
\tfrac{1}{\sqrt{2}}(\ket{00}-\ket{10})
\]
\[
=\bigl(\tfrac{\ket{0}-\ket{1}}{\sqrt{2}}\bigr)\otimes\ket{0}=\ket{-}\otimes\ket{0}
\]
\[
\xrightarrow{H\otimes I}
\ket{1}\otimes\ket{0}=\ket{10}
\]

\textbf{Caso \(b_1=1,b_2=1\).}
\[
\beta_{00}=\tfrac{1}{\sqrt{2}}(\ket{00}+\ket{11})
\]
\[
\xrightarrow{X^{1}\otimes I}
\tfrac{1}{\sqrt{2}}(\ket{10}+\ket{01})
\]
\[
\xrightarrow{Z^{1}\otimes I}
\tfrac{1}{\sqrt{2}}(-\ket{10}+\ket{01})
\]
\[
\xrightarrow{\mathrm{CNOT}}
\tfrac{1}{\sqrt{2}}(-\ket{11}+\ket{01})
\]
\[
=\bigl(\tfrac{\ket{0}-\ket{1}}{\sqrt{2}}\bigr)\otimes\ket{1}=\ket{-}\otimes\ket{1}
\]
\[
\xrightarrow{H\otimes I}
\ket{1}\otimes\ket{1}=\ket{11}
\]

\newpage
\section*{Algoritmo}
El algoritmo de codificación superdensa empieza en un estado $\beta_{00}$, donde Alice tiene el primer qubit y Bob el segundo. Luego de aplicar la primer compuerta, Alice envía su qubit a Bob. Alice quiere enviar los bits $b_1b_2$ clásicos a Bob, el circuito es de la forma:

\[
	\Qcircuit @R=1em @C=1em{
	\lstick{Alice} & \qw & \gate{Z^{b_1}X^{b_2}} & \qw & \ctrl{1} & \gate{H} & \meter\\
	\lstick{Bob}   & \qw & \qw                   & \qw & \targ    & \qw & \meter\\
	}
\]


\subsection*{Lambda cálculo cuántico a control clásico}

El término que implementa el algoritmo en el lambda cuántico a control clásico es el siguiente:

\begin{align*}
    (\lambda q_1. \lambda q_2. \lambda b_1. \lambda b_2. &\\
        & \textbf{ let } \langle a, b \rangle = CNOT (& & \textbf{ let } \langle xq_1, q_2 \rangle = \textbf{ if } b_2 \textbf{ then } \langle X q_1, q_2 \rangle \textbf{ else }  \langle q_1, q_2 \rangle \\
        &                               & & \textbf{ in } \textbf{ if } b_1 \textbf{ then } \langle Z xq_1, xq_2 \rangle \textbf{ else } \langle xq_1, xq_2 \rangle ) \\
        & \textbf{ in } \langle \mathrm{\text{meas }} (H a), \mathrm{\text{meas }} b \rangle) \\
    \text{algo que construye } \beta_{00}
\end{align*}




Su tipo es ??


\end{document}
