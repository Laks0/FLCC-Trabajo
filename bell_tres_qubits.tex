\subsection*{III. 15 Dar un circuito que genere el estado $\frac{1}{\sqrt{2}}|000\rangle + \frac{1}{\sqrt{2}}|111\rangle$ a partir de la entrada $|000\rangle$}

Funciona el siguiente circuito:

\[
	\Qcircuit @R=1em @C=1em{
		\lstick{\ket{0}} & \gate{H} & \ctrl{1} & \ctrl{2} & \qw \\
		\lstick{\ket{0}} & \qw      & \targ    & \qw      & \qw \\
		\lstick{\ket{0}} & \qw      & \qw      & \targ    & \qw \\
	}
\]

Si analizamos la traza vemos que, comenzando con la entrada \ket{000} llegamos al estado deseado.

\def\hr{\frac{1}{\sqrt{2}}}

\[
	\ket{000} \xrightarrow{H\otimes I \otimes I} (\hr\ket{0}+\hr\ket{1}) \ket{00} = \hr (\ket{000} + \ket{100})
\]
\[
	\xrightarrow{CNOT(1,2)} \hr (\ket{000}+\ket{110}) \xrightarrow{CNOT(1,3)} \hr (\ket{000}+\ket{111})
\]
