\subsection*{Cálculo $\mathcal{L}^\mathbb{C}$}



% Thus, we have the multiplicative truth $\mathbf{1}$, the multiplicative implication $\multimap$, the multiplicative conjunction $\otimes$, 
% the additive truth $\top$, the additive falsehood $\mathbf{0}$, the additive conjunction~\&{}, and the additive disjunction~\oplus.


% ===============================
% Reduction Rules (LaTeX-ready)
% ===============================

% --- 1. Unit elimination ---
\[
\delta_{1}(a\cdot \star , t) \;\longrightarrow\; a \bullet t
\]

% --- 2. Beta reduction ---
\[
(\lambda x.\, t)\, u \;\longrightarrow\; (u/x)\, t
\]

% --- 3. Tensor elimination ---
\[
\delta_{\otimes}(u \otimes v,\; xy.\, w) 
\;\longrightarrow\; (u/x,\; v/y)\, w
\]

% --- 4. With-elimination (left) ---
\[
\delta_{1\&}(\langle t, u\rangle,\; x.\, v) 
\;\longrightarrow\; (t/x)\, v
\]

% --- 5. With-elimination (right) ---
\[
\delta_{2\&}(\langle t, u\rangle,\; x.\, v) 
\;\longrightarrow\; (u/x)\, v
\]

% --- 6. Sum-elimination (inl) ---
\[
\delta_{\oplus}(\mathrm{inl}(t),\; x.\, v,\; y.\, w) 
\;\longrightarrow\; (t/x)\, v
\]

% --- 7. Sum-elimination (inr) ---
\[
\delta_{\oplus}(\mathrm{inr}(u),\; x.\, v,\; y.\, w) 
\;\longrightarrow\; (u/y)\, w
\]

% --- 8. Additivity: unit ---
\[
a\cdot\star \;+\; b\cdot\star 
\;\longrightarrow\; (a+b)\cdot\star
\]

% --- 9. Additivity: lambda ---
\[
(\lambda x.\, t) + (\lambda x.\, u) 
\;\longrightarrow\; \lambda x.\, (t+u)
\]

% --- 10. Additivity: tensor delta ---
\[
\delta_{\otimes}(t+u,\; xy.\, v) 
\;\longrightarrow\; 
\delta_{\otimes}(t,\; xy.\, v) + \delta_{\otimes}(u,\; xy.\, v)
\]

% --- 11. Additivity: unit pair ---
\[
\langle\,\rangle + \langle\,\rangle 
\;\longrightarrow\; \langle\,\rangle
\]

% --- 12. Additivity: pair ---
\[
\langle t, u\rangle + \langle v, w\rangle
\;\longrightarrow\; \langle t+v,\; u+w\rangle
\]

% --- 13. Additivity: sum delta ---
\[
\delta_{\oplus}(t+u,\; x.\, v,\; y.\, w)
\;\longrightarrow\; 
\delta_{\oplus}(t,\; x.\, v,\; y.\, w)
\;+\;
\delta_{\oplus}(u,\; x.\, v,\; y.\, w)
\]

% --- 14. Scalar product: unit ---
\[
a \bullet (b\cdot\star) 
\;\longrightarrow\; (a\times b)\cdot\star
\]

% --- 15. Scalar product: lambda ---
\[
a \bullet (\lambda x.\, t) 
\;\longrightarrow\; \lambda x.\, (a\bullet t)
\]

% --- 16. Scalar product: tensor delta ---
\[
\delta_{\otimes}(a\bullet t,\; xy.\, v)
\;\longrightarrow\; 
a\bullet \delta_{\otimes}(t,\; xy.\, v)
\]

% --- 17. Scalar product: empty pair ---
\[
a\bullet \langle\,\rangle \;\longrightarrow\; \langle\,\rangle
\]

% --- 18. Scalar product: pair ---
\[
a\bullet \langle t, u\rangle 
\;\longrightarrow\; 
\langle a\bullet t,\; a\bullet u\rangle
\]

% --- 19. Scalar product: sum delta ---
\[
\delta_{\oplus}(a\bullet t,\; x.\, v,\; y.\, w)
\;\longrightarrow\; 
a\bullet \delta_{\oplus}(t,\; x.\, v,\; y.\, w)
\]

% --- 20. Sum-elimination (first) for ⊙ ---
\[
\delta_{1\odot}([t,u],\; x.\, v)
\;\longrightarrow\; (t/x)\, v
\]

% --- 21. Sum-elimination (second) for ⊙ ---
\[
\delta_{2\odot}([t,u],\; x.\, v)
\;\longrightarrow\; (u/x)\, v
\]

% --- 22. Sum-elimination (inl-style) for ⊙ ---
\[
\delta_{\odot}([t,u],\; x.\, v,\; y.\, w)
\;\longrightarrow\; (t/x)\, v
\]

% --- 23. Sum-elimination (inr-style) for ⊙ ---
\[
\delta_{\odot}([t,u],\; x.\, v,\; y.\, w)
\;\longrightarrow\; (u/y)\, w
\]

% --- 24. Additivity: list ---
\[
[t,u] + [v,w]
\;\longrightarrow\; [t+v,\; u+w]
\]

% --- 25. Scalar product: list ---
\[
a \bullet [t,u]
\;\longrightarrow\; [a\bullet t,\; a\bullet u]
\]




Usamos la construcción de matrices dada en el paper para $\mathbb{C}^{2\times 2}$ de la forma:

$$
\left(
\begin{matrix}
	a & b \\
	c & d
\end{matrix}
\right)
= \lambda x. (\delta_\odot^1(x, y.\delta_\top(y, [a.\star, b.\star]) + \delta_\odot^2(x, z.\delta_\top(z,[c.\star,d.\star]))))$$

Que asumimos como correcta y usamos para describir:

$$
H = \left(
	\begin{matrix}
	\hr & \hr \\
	\hr & -\hr
\end{matrix}
\right) ;
X = \left(
	\begin{matrix}
	0 & 1 \\
	1 & 0
\end{matrix}
\right) ;
Z = \left(
	\begin{matrix}
	1 & 0 \\
	0 & -1
\end{matrix}
\right) ;
I = \left(
	\begin{matrix}
	1 & 0 \\
	0 & 1
\end{matrix}
\right) ;
\emptyset = \left(
	\begin{matrix}
	0 & 0 \\
	0 & 0
\end{matrix}
\right)
$$

Y usamos esas construcciones para definir:

\[
	Z^{b_1} = \lambda b_1. (\lambda x. \delta_\vee(b_1, x, Z\ x)) b_1
\]

\[
	X^{b_2} = \lambda b_2. (\lambda x. \delta_\vee(b_2, x, X\ x)) b_2
\]
\[
	H\otimes I = \lambda x. [\delta_\odot^1(x, y. H\ y), \delta_\odot^2(x, z.z)]
\]

Usando la representación de que un bit es de tipo $\top \vee \top$ y $\mathbf{0} = inl(1.\star); \mathbf{1} = inr(1.\star)$.

Usando la construcción recursiva del paper también podemos construir:
\[
	\text{CNOT } = \left(\begin{matrix}
			I & \emptyset \\
			\emptyset & X
	\end{matrix}\right)
\]

$$\overline{\ket{00}} = [[1.\star, 0.\star], [0.\star, 0.\star]]
;
\overline{\ket{11}} = [[0.\star, 0.\star], [0.\star, 1.\star]]$$

\[
	\pi = \lambda q. \langle\delta_\odot^1(q, x.\delta_\odot(x, y.\mathbf{0}, z.\mathbf{1})), \delta_\odot^2(q, x.\delta_\odot(x, y.\mathbf{0}, z.\mathbf{1}) \rangle
\]

\[
(\lambda q.\ \lambda b_{1}.\ \lambda b_{2}.\ \pi (H \otimes I)\,\text{CNOT}\!\left(Z^{b_{1}}\!\left(X^{b_{2}}\, q\right)\right))
\ \hr\!\cdot(\overline{\ket{00}}+\overline{\ket{11}})
\]



---------------------------------------------
\newpage

Derivación para $b_1 = \text{inr}(1.\star), b_2 = \text{inr}(1.\star)$.
\[
    (\lambda q. \lambda b_1. \lambda b_2. \pi (H\otimes I) \text{CNOT} (Z^{b_1} (X^{b_2} q))) \Psi \ \text{inr}(1.\star) \ \text{inr}(1.\star)
\]
%
\begin{align*}
&\longrightarrow_{}
\pi (H\otimes I)\,\text{CNOT}\bigl((Z^{\text{inr}(1.\star)}\otimes I)((X^{\text{inr}(1.\star)}\otimes I)\Psi)\bigr)\\
&\longrightarrow_{}
\pi (H\otimes I)\,\text{CNOT}\bigl((Z\otimes I)((X\otimes I)\Psi)\bigr)\\
&\longrightarrow_{}
\pi (H\otimes I)\,\text{CNOT}\!\left((Z\otimes I)\!\left((X\otimes I)\!\left(\hr(\overline{\ket{00}}+\overline{\ket{11}})\right)\right)\right)\\
&\longrightarrow_{}
\hr\,\pi (H\otimes I)\,\text{CNOT}\bigl((Z\otimes I)((X\otimes I)\overline{\ket{00}}+(X\otimes I)\overline{\ket{11}})\bigr)\\
&\longrightarrow_{}
\hr\,\pi (H\otimes I)\,\text{CNOT}\bigl((Z\otimes I)(\overline{\ket{10}}+\overline{\ket{01}})\bigr)\\
&\longrightarrow_{}
\hr\,\pi (H\otimes I)\,\text{CNOT}\bigl((Z\otimes I)\overline{\ket{10}}+(Z\otimes I)\overline{\ket{01}}\bigr)\\
&\longrightarrow_{}
\hr\,\pi (H\otimes I)\,\text{CNOT}\bigl(-\overline{\ket{10}}+\overline{\ket{01}}\bigr)\\
&\longrightarrow_{}
\hr\,\pi (H\otimes I)\bigl(-\text{CNOT}\,\overline{\ket{10}}+\text{CNOT}\,\overline{\ket{01}}\bigr)\\
&\longrightarrow_{}
\hr\,\pi (H\otimes I)\bigl(-\overline{\ket{11}}+\overline{\ket{01}}\bigr)\\
&\longrightarrow_{}
\hr\,\pi\bigl(-(H\otimes I)\overline{\ket{11}}+(H\otimes I)\overline{\ket{01}}\bigr)\\
&\longrightarrow_{}
\hr\,\pi\!\left(
-\hr(\overline{\ket{01}}-\overline{\ket{11}})
+\hr(\overline{\ket{01}}+\overline{\ket{11}})
\right)\\
&\longrightarrow_{}
(\hr\cdot\hr)\,\pi\bigl(-\overline{\ket{01}}+\overline{\ket{11}}
+\overline{\ket{01}}+\overline{\ket{11}}\bigr)\\
&\longrightarrow_{}
\frac12\,\pi(2\,\overline{\ket{11}})\\
&\longrightarrow_{}
\pi(\overline{\ket{11}})\\
&\longrightarrow_{}
\langle \mathbf{1},\mathbf{1}\rangle
\end{align*}