\subsection*{Cálculo $\mathcal{L}^\mathbb{C}$}

Usamos la construcción de matrices dada en el paper para $\mathbb{C}^{2\times 2}$ de la forma:

$$
\left(
\begin{matrix}
	a & b \\
	c & d
\end{matrix}
\right)
= \lambda x. (\delta_\odot^1(x, y.\delta_\top(y, [a.\star, b.\star]) + \delta_\odot^2(x, z.\delta_\top(z,[c.\star,d.\star]))))$$

Que asumimos como correcta y usamos para describir:

$$
H = \left(
	\begin{matrix}
	\hr & \hr \\
	\hr & -\hr
\end{matrix}
\right) ;
X = \left(
	\begin{matrix}
	0 & 1 \\
	1 & 0
\end{matrix}
\right) ;
Z = \left(
	\begin{matrix}
	1 & 0 \\
	0 & -1
\end{matrix}
\right)
$$

Y usamos esas construcciones para definir:

\[
	Z^{b_1} = \lambda b_1. \lambda x. \delta_\vee(b_1, x, Z\ x)
;
	X^{b_2} = \lambda b_2. \lambda x. \delta_\vee(b_2, x, X\ x)
\]
\[
	H_1 = \lambda x. [\delta_\odot^1(x, y. H\ y), \delta_\odot^2(x, z.z)]
\]

Usando la representación de que un bit es de tipo $\top \vee \top$ y $\mathbf{0} = inl(1.\star); \mathbf{1} = inr(1.\star)$.

\begin{align*}
	& \lambda b_1. \lambda b_2 \\
	& \hr\cdot(\overline{\ket{00}}+\overline{\ket{11}})
\end{align*}
