\subsection*{Lambda cálculo cuántico a control clásico}

El término que implementa el algoritmo en el lambda cuántico a control clásico es el siguiente:

\begin{align*}
    (\lambda b_1. \lambda b_2. &\\
				&\textbf{let } \langle q_1, q_2 \rangle = EPR \star\\
				&\textbf{in let } \langle a, b \rangle = CNOT (& & \textbf{ let } \langle x_1, x_2 \rangle = \textbf{ if } b_2 \textbf{ then } \langle X q_1, q_2 \rangle \textbf{ else }  \langle q_1, q_2 \rangle \\
        &                               & & \textbf{ in } \textbf{ if } b_1 \textbf{ then } \langle Z x_1, x_2 \rangle \textbf{ else } \langle x_1, x_2 \rangle ) \\
        & \textbf{in } \langle \mathrm{\text{meas }} (H a), \mathrm{\text{meas }} b \rangle)
\end{align*}
donde $EPR =\lambda x. CNOT \langle H (\text{new } 0), \text{new } 0 \rangle$

El tipo del término es [\ket{}, t]: $!bit \multimap !bit \multimap qubit \otimes qubit$

Veamos su reducción paso a paso cuando es ejecutado con $b_1 =  b_2 = 1$.


\begin{align*}
    [\ket{}, 
    & & (\lambda b_1. \lambda b_2. &\\
	& & 			&\textbf{let } \langle q_1, q_2 \rangle = EPR \star\\
	& & 			&\textbf{in let } \langle a, b \rangle = CNOT (& & \textbf{ let } \langle x_1, x_2 \rangle = \textbf{ if } b_2 \textbf{ then } \langle X q_1, q_2 \rangle \textbf{ else }  \langle q_1, q_2 \rangle \\
    & &     &                               & & \textbf{ in } \textbf{ if } b_1 \textbf{ then } \langle Z x_1, x_2 \rangle \textbf{ else } \langle x_1, x_2 \rangle ) \\
    & &     & \textbf{in } \langle \mathrm{\text{meas }} (H a), \mathrm{\text{meas }} b \rangle) \\
    & & 1 \; 1 ]
\end{align*} 

$\rightarrow$

\begin{align*}
    [\ket{}, 
    & & (\lambda b_2. &\\
	& & 			&\textbf{let } \langle q_1, q_2 \rangle = EPR \star\\
	& & 			&\textbf{in let } \langle a, b \rangle = CNOT (& & \textbf{ let } \langle x_1, x_2 \rangle = \textbf{ if } b_2 \textbf{ then } \langle X q_1, q_2 \rangle \textbf{ else }  \langle q_1, q_2 \rangle \\
    & &     &                               & & \textbf{ in } \textbf{ if } 1 \textbf{ then } \langle Z x_1, x_2 \rangle \textbf{ else } \langle x_1, x_2 \rangle ) \\
    & &     & \textbf{in } \langle \mathrm{\text{meas }} (H a), \mathrm{\text{meas }} b \rangle) \\
    & & 1 ]
\end{align*} 

$\rightarrow$

\begin{align*}
    [\ket{}, 
    &\textbf{let } \langle q_1, q_2 \rangle = EPR \star\\
	&\textbf{in let } \langle a, b \rangle = CNOT (& & \textbf{ let } \langle x_1, x_2 \rangle = \textbf{ if } 1 \textbf{ then } \langle X q_1, q_2 \rangle \textbf{ else }  \langle q_1, q_2 \rangle \\
    &                               & & \textbf{ in } \textbf{ if } 1 \textbf{ then } \langle Z x_1, x_2 \rangle \textbf{ else } \langle x_1, x_2 \rangle ) \\
    & \textbf{in } \langle \mathrm{\text{meas }} (H a), \mathrm{\text{meas }} b \rangle) ]
\end{align*} 


$\rightarrow^*$ la reducción de $EPR \star$ la vimos en clase y en varios pasos lo siguiente, donde $p_1$ es una referencia el qubit de Alice y $p_2$ al de bob:

\begin{align*}
    [ \ket{\frac{1}{\sqrt{2}}(\ket{00}+\ket{11})}, 
    &\textbf{let } \langle q_1, q_2 \rangle = \langle p_1, p_2 \rangle \\
	&\textbf{in let } \langle a, b \rangle = CNOT (& & \textbf{ let } \langle x_1, x_2 \rangle = \textbf{ if } 1 \textbf{ then } \langle X q_1, q_2 \rangle \textbf{ else }  \langle q_1, q_2 \rangle \\
    &                               & & \textbf{ in } \textbf{ if } 1 \textbf{ then } \langle Z x_1, x_2 \rangle \textbf{ else } \langle x_1, x_2 \rangle ) \\
    & \textbf{in } \langle \mathrm{\text{meas }} (H a), \mathrm{\text{meas }} b \rangle ]
\end{align*} 


$\rightarrow$

\begin{align*}
    [ \ket{\frac{1}{\sqrt{2}}(\ket{00}+\ket{11})}, 
	&\textbf{let } \langle a, b \rangle = CNOT (& & \textbf{ let } \langle x_1, x_2 \rangle = \textbf{ if } 1 \textbf{ then } \langle X p_1, p_2 \rangle \textbf{ else }  \langle p_1, p_2 \rangle \\
    &                               & & \textbf{ in } \textbf{ if } 1 \textbf{ then } \langle Z x_1, x_2 \rangle \textbf{ else } \langle x_1, x_2 \rangle ) \\
    & \textbf{in } \langle \mathrm{\text{meas }} (H a), \mathrm{\text{meas }} b \rangle ]
\end{align*} 


$\rightarrow$

\begin{align*}
    [ \ket{\frac{1}{\sqrt{2}}(\ket{00}+\ket{11})}, 
	&\textbf{let } \langle a, b \rangle = CNOT (& & \textbf{ let } \langle x_1, x_2 \rangle = \langle X p_1, p_2 \rangle \\
    &                               & & \textbf{ in } \textbf{ if } 1 \textbf{ then } \langle Z x_1, x_2 \rangle \textbf{ else } \langle x_1, x_2 \rangle ) \\
    & \textbf{in } \langle \mathrm{\text{meas }} (H a), \mathrm{\text{meas }} b \rangle ]
\end{align*} 


$\rightarrow$

\begin{align*}
    [ \ket{\frac{1}{\sqrt{2}}(\ket{10}+\ket{01})}, 
	&\textbf{let } \langle a, b \rangle = CNOT (& & \textbf{ let } \langle x_1, x_2 \rangle = \langle p_1, p_2 \rangle \\
    &                               & & \textbf{ in } \textbf{ if } 1 \textbf{ then } \langle Z x_1, x_2 \rangle \textbf{ else } \langle x_1, x_2 \rangle ) \\
    & \textbf{in } \langle \mathrm{\text{meas }} (H a), \mathrm{\text{meas }} b \rangle ]
\end{align*} 

$\rightarrow$

\begin{align*}
    [ \ket{\frac{1}{\sqrt{2}}(\ket{10}+\ket{01})}, 
	&\textbf{let } \langle a, b \rangle = CNOT (\textbf{ if } 1 \textbf{ then } \langle Z p_1, p_2 \rangle \textbf{ else } \langle p_1, p_2 \rangle ) \\
    & \textbf{in } \langle \mathrm{\text{meas }} (H a), \mathrm{\text{meas }} b \rangle ]
\end{align*} 

$\rightarrow$

\begin{align*}
    [ \ket{\frac{1}{\sqrt{2}}(\ket{10}+\ket{01})}, 
	&\textbf{let } \langle a, b \rangle = CNOT (\langle Z p_1, p_2 \rangle) \\
    & \textbf{in } \langle \mathrm{\text{meas }} (H a), \mathrm{\text{meas }} b \rangle ]
\end{align*} 

$\rightarrow$

\begin{align*}
    [ \ket{\frac{1}{\sqrt{2}}(-\ket{10}+\ket{01})}, 
	&\textbf{let } \langle a, b \rangle = CNOT (\langle p_1, p_2 \rangle) \\
    & \textbf{in } \langle \mathrm{\text{meas }} (H a), \mathrm{\text{meas }} b \rangle ]
\end{align*} 

$\rightarrow$

\begin{align*}
    [ \ket{\frac{1}{\sqrt{2}}(-\ket{11}+\ket{01})}, 
	&\textbf{let } \langle a, b \rangle = \langle p_1, p_2 \rangle \\
    & \textbf{in } \langle \mathrm{\text{meas }} (H a), \mathrm{\text{meas }} b \rangle ]
\end{align*} 

$\rightarrow$ dado que $\ket{\frac{1}{\sqrt{2}}(-\ket{11}+\ket{01})}$ es igual a $\ket{-} \otimes \ket{1}$

\begin{align*}
    [ \ket{-} \otimes \ket{1}, \langle \mathrm{\text{meas }} (H p_1), \mathrm{\text{meas }} p_2 \rangle ]
\end{align*} 

$\rightarrow$

\begin{align*}
    [ \ket{1} \otimes \ket{1}, \langle \mathrm{\text{meas }} p_1, \mathrm{\text{meas }} p_2 \rangle ]
\end{align*} 

Dado que el estado es $\ket{11}$, al medir ambos qubits con probabilidad 1 se va a obtener un 1. Por ende no escribimos los caminos que surgirían de que al medir se obtenga un 0.

\begin{align*}
    [ \ket{1} \otimes \ket{1}, \langle \mathrm{\text{meas }} p_1, \mathrm{\text{meas }} p_2 \rangle ]
\end{align*} 

$\rightarrow$

\begin{align*}
    [ \ket{1} \otimes \ket{1}, \langle 1, \mathrm{\text{meas }} p_2 \rangle ]
\end{align*} 

$\rightarrow$

\begin{align*}
    [ \ket{1} \otimes \ket{1}, \langle 1, 1 \rangle ]
\end{align*} 