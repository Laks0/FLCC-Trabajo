\def\r{\longrightarrow_R}
\def\rs{\longrightarrow^{\star}_R}

\subsection*{Ejercicio 8: demostrar el siguiente lema} 

\begin{myblock}{Lema 2.7}
	\begin{enumerate}
		\item Si $\r$ satisface la propiedad del diamante, entonces es Church-Rosser.
		\item Si $\r$ es Church-Rosser, entonces tiene formas normales únicas.
	\end{enumerate}
\end{myblock}

\subsubsection*{Lema 2.7.1}

Sea $\r$ que satisface la propiedad del diamante, es decir que, si $t \r r_1$ y $t \r r_2$ entonces existe un $s$ tal que $r_1 \r s$ y $r_2 \r s$. Queremos ver que si $t \rs r_1$ y $t \rs r_2$ entonces existe un $s$ tal que $r_1 \rs s$ y $r_2 \rs s$.

Sean $t, r_1, r_2$ tales que $t \rs r_1, t \rs r_2$. Si $t \rs r_1$ existe una secuencia $\alpha_1, \alpha_2, \ldots, \alpha_{k-1}$ tal que $t \r \alpha_1 \r \alpha_2 \r \ldots \r \alpha_{k-1} \r r_1$, simétricamente para $r_2$ existe una secuencia $\beta_1, \ldots, \beta_{l-1}$. Notamos esto $t \r^k r_1, t \r^l r_2$. Veamos por inducción en $k$ y $l$ que es Church-Rosser.

Para $l,r \leq 1$ se sostiene que es CR pues es la propiedad del diamante.

Veamos inductivamente que se sostiene para $l+1, r+1 \leq n+1$. \textbf{Hipótesis inductiva:} Si $t \r^k r_1, t \r^l r_2$ entonces existe un $s$ tal que $r_1 \r^l s, r_2 \r^k s$.

Sean $t,r_1,r_2$ tales que $t \r^{k+1} r_1, t \r^{l+1} r_2$. 
Por definición de $\r^{\_+1}$ existen $\alpha$ y $\beta$ tales que $t \r^k \alpha \r r_1$ y $t \r^l \beta \r r_2$. 
Luego, por hipótesis inductiva tenemos que existe un $\gamma$ tal que $\alpha \r^l \gamma$ y $\beta \r^k \gamma$. 
Ahora también por HI sabemos que existen $\delta_1, \delta_2$ tal que $r_1 \r^l \delta_1, r_2 \r^k \delta_2, \gamma \r \delta_1, \gamma \r \delta_2$. 
Finalmente, por propiedad de diamante con $\gamma, \delta_1, \delta_2$ vemos que existe $s$ tal que $\delta_1 \r s, \delta_2 \r s$.

Uniendo todo esto tenemos que $r_1 \r^l \delta_1 \r s$ y $r_2 \r^k \delta_2 \r s$ por lo tanto $r_1 \r^{l+1} s$ y $r_2 \r^{k+1} s$ que era lo que buscábamos. Abajo hay un esquema de la demostración.

\begin{center}
	\begin{forest}
		for tree = {edge={dotted, -Straight Barb}}
		[$t$
			[$\alpha$, edge label = {node[midway, left, font=\scriptsize]{k}}, name=a
				[$r_1$, edge={solid}
					[$\delta_1$, name=d1, edge label = {node[midway, left, font=\scriptsize]{l}}]
				]
			]
			[, no edge[$\gamma$, name=g, no edge
					[, no edge[$s$, name=s, no edge]]
			]]
			[$\beta$, edge label = {node[midway, right, font=\scriptsize]{l}}, name=b
				[$r_2$, edge={solid}
					[$\delta_2$, name=d2, edge label = {node[midway, right, font=\scriptsize]{k}}]
				]
			]
		]
		\draw[-Straight Barb, dotted] (a) -- (g) node[midway, right, font=\scriptsize] {l};
		\draw[-Straight Barb, dotted] (b) -- (g) node[midway, left, font=\scriptsize] {k};
		\draw[-Straight Barb] (g) -- (d1);
		\draw[-Straight Barb] (g) -- (d2);
		\draw[-Straight Barb] (d1) -- (s);
		\draw[-Straight Barb] (d2) -- (s);
	\end{forest}
\end{center}

\subsubsection*{Lema 2.7.2}

Sea $\r$ Church-Rosser, $r_1, r_2$ en forma normal, y $t$ tal que $t \rs r_1, t \rs r_2$ queremos probar que $r_1 = r_2$.

Como $\r$ es Church-Rosser y $t \rs r_1, t \rs r_2$, sabemos que existe un $s$ tal que $r_1 \rs s$ y $r_2 \rs s$.

Dado que $r_1$ está en forma normal, no existe ningún $e$ tal que $r_1 \r e$, por lo que el único elemento tal que $r_1 \rs s$ es $r_1$. Por lo tanto $s = r_1$. Análogamente ocurre con $r_2$, por lo tanto $r_2 = s$ y por ende $r_1 = r_2$.
